\documentclass[12pt, a4paper]{report}

% 页面设置
\usepackage[top=1in, bottom=1in, left=1in, right=1in]{geometry}
\usepackage{ctex}
\usepackage{amsmath}
\usepackage{graphicx}
\usepackage{subfigure}
% \usepackage{float}%提供float浮动环境
\usepackage{booktabs}%提供命令
\usepackage{tocbibind}
\usepackage{color}
\usepackage[dvipsnames]{xcolor}
\usepackage{colortbl}
\definecolor{mygray}{gray}{0.15}
\definecolor{ccr}{RGB}{0,100,20}

\usepackage{hyperref}
\hypersetup{hypertex=true,
    colorlinks=true,
    linkcolor=mygray,
    anchorcolor=ccr,
    citecolor=ccr,
    urlcolor=ccr}

\usepackage{titlesec} % 导入 titlesec 宏包
% 调整章节标题与顶部的距离
\titlespacing*{\chapter}{0pt}{-10pt}{20pt}

% 定义章节标题格式
\titleformat{\chapter}[block]{\centering\Large\bfseries}{第\chinese{chapter}章}{1em}{}

\titleformat{\section}
  {\large\bfseries} % 修改字号和字体样式
  {\thesection} % 章节编号
  {1em} % 编号与标题之间的间距
  {} % 标题前的格式化内容

% 导入 fancyhdr 宏包
\usepackage{fancyhdr}  % 用于设置页眉页脚
% 设置页面风格
\pagestyle{fancy}
% 定义页眉和页脚
\fancyhf{}  % 清空默认的页眉和页脚
% 设置页眉内容
% -------------------------
% \fancyhead[L]{第~\chinese{chapter}~章}    % 左侧页眉 - 显示当前章节编号
% \fancyhead[C]{深度神经网络的视觉语义特征可解释技术研究}          % 中间页眉 - 显示论文题目
% \fancyhead[R]{\rightmark}               % 右侧页眉 - 显示作者名字
% -------------------------
\fancyhead[C]{\ifodd\value{page} \else xxx研究 \fi}
\fancyhead[LO]{\ifodd\value{page} 第~\chinese{chapter}~章 \else \fi}
\fancyhead[RO]{\ifodd\value{page} 温州大学硕士学位论文 \else \fi}
% ----------------------
% 自定义页眉:判断是否为章节页面


% 设置页脚内容
\fancyfoot[C]{\thepage}             % 中间页脚 - 显示页码

\title{毕业论文题目}  % 默认主标题(可以在 titlepage.tex 中覆盖)
\author{Zewen Yu}      % 默认作者(可以在 titlepage.tex 中覆盖)
\date{\today}        % 默认日期(可以在 titlepage.tex 中覆盖)

\begin{document}

% 引用封面
\begin{titlepage}
    \centering
    % 插入图片,确保图片文件已上传到 Overleaf 项目中
    \

    \
    
    \includegraphics[width=0.11\textwidth]{title_img/logo.png}
    \includegraphics[width=0.3\textwidth]{title_img/wzu_xiaoming.jpeg} \\[0.5cm]
    {\Huge 硕士毕业论文} \\[1.5cm]
    \begin{table}[h]
    \centering
    \begin{tabular}{lp{8cm}}
    \songti{\Large{题目:}} & \heiti{\Large{xxxx研究}}\\
    \end{tabular}
    \end{table}
    \renewcommand{\arraystretch}{1.5} % 将表格的行高设置为原来的1.5倍
    \begin{table}[h]\large
    \centering
    \begin{tabular}{lc}
    \makebox[0.08\textwidth][l]{\heiti{姓\quad\quad 名:}} & \makebox[0.35\textwidth][c]{\fangsong{于泽文}} \\
    \cline{2-2}
    % \heiti{姓\quad\quad 名:} & \fangsong{于泽文} \\
    \heiti{学\quad\quad 号:} & \fangsong{xxxxxxxxx} \\
    \cline{2-2}
    \heiti{院\quad\quad 系:} & \fangsong{计算机与人工智能} \\
    \cline{2-2}
    \heiti{专\quad\quad 业:} & \fangsong{计算机科学与技术} \\
    \cline{2-2}
    \heiti{研究方向:} & \fangsong{深度神经网络和计算机视觉} \\
    \cline{2-2}
    \heiti{导\quad\quad 师:} & \fangsong{} \\
    \cline{2-2}
    \end{tabular}
    \end{table}
    \vfill
    {\large \today}
\end{titlepage}
% 引用摘要
% 中文摘要
\chapter*{xxxx研究\\ \vspace{1cm}摘要}
\addcontentsline{toc}{chapter}{摘要}  % 将摘要添加到目录中

\textbf{关键词:AAA, BBB, CCC}
\bigskip % 添加段落间距

% 英文摘要

\chapter*{Research on xxxxx\\ \vspace{1cm}Abstract}
\addcontentsline{toc}{chapter}{Abstract}  % 将英文摘要添加到目录中

\textbf{Keywords: AAA, BBB, CCC}
\tableofcontents  % 自动生成目录

% 引用正文
\cleardoublepage
\chapter{绪论}
\thispagestyle{fancy}  % 强制设置该章第一页的页眉样式
\section{研究背景和意义}

\section{国内外研究现状}


\newpage
\section{研究目的和主要工作}

\newpage
\section{论文的组织结构}


\chapter{相关技术}
\thispagestyle{fancy}  % 强制设置该章第一页的页眉样式
\section{AAAA}

\section{BBB}

\section{CCC}


\chapter{图片使用}
\thispagestyle{fancy}  % 强制设置该章第一页的页眉样式
\section{单个图片}
\begin{figure}[h]
    \centering
    \includegraphics[width=0.5\linewidth]{images/testforpic.pdf}
    \caption{Caption}
    \label{fig:enter-label}
\end{figure}

\section{多个图片}
\begin{figure}[h]
    \centering
    \subfigure[]{\includegraphics[width=0.3\linewidth]{images/testforpic.pdf}}
    \subfigure[]{\includegraphics[width=0.3\linewidth]{images/testforpic.pdf}}
    \caption{Caption}
    \label{fig:enter-label2}
\end{figure}

\chapter{表格使用}
\thispagestyle{fancy}  % 强制设置该章第一页的页眉样式
\section{三线表}
\begin{table}[h]
    \centering
    \begin{tabular}{c|c|c|c|c}
        \toprule
        A & B & B & B & B\\
        \midrule
        C & D & & & \\
        \midrule
        E & F & & & \\
        \bottomrule
    \end{tabular}
    \caption{Caption}
    \label{tab:my_label}
\end{table}

\chapter{公式使用}
\section{单行公式}
\begin{equation}\label{eqn:5-1}
    A = B + \sum_{i=1}^{n}C_{i}
\end{equation}
\section{多行公式}
\begin{subequations}
\begin{align}
    A &= C + E \label{eqn:5-2a}\\
      B  &= F + \sum_{i=1}^{n} G_{i}\label{eqn:5-2b}
\end{align}
\end{subequations}
    
\chapter{结论}
\thispagestyle{fancy}
本文总结了研究发现,提出了相关建议...


\bibliographystyle{gbt7714-2005}
\bibliography{references}

\chapter*{致~谢}

\chapter*{攻读硕士期间的科研成果}
\end{document}
